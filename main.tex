\documentclass[12pt,article,oneside,a4paper]{memoir}

%% Packages
%% ========
\usepackage{graphicx}
\usepackage{titlesec}
\usepackage{wrapfig}
	
\setcounter{secnumdepth}{4}

\titleformat{\paragraph}
{\normalfont\normalsize\bfseries}{\theparagraph}{1em}{}
\titlespacing*{\paragraph}
{0pt}{3.25ex plus 1ex minus .2ex}{1.5ex plus .2ex}

%% many common packages
%% LaTeX Font encoding -- DO NOT CHANGE
\usepackage[OT1]{fontenc}

%% Babel provides support for languages.  'english' uses British
%% English hyphenation and text snippets like "Figure" and
%% "Theorem". Use the option 'ngerman' if your document is in German.
%% Use 'american' for American English.  Note that if you change this,
%% the next LaTeX run may show spurious errors.  Simply run it again.
%% If they persist, remove the .aux file and try again.
\usepackage[english]{babel}

%% Input encoding 'utf8'. In some cases you might need 'utf8x' for
%% extra symbols. Not all editors, especially on Windrows, are UTF-8
%% capable, so you may want to use 'latin1' instead.
\usepackage[utf8]{inputenc}

%% This changes default fonts for both text and math mode to use Herman Zapfs
%% excellent Palatino font.  Do not change this.
\usepackage[sc]{mathpazo}


%% We unfortunately need this for the Rules chapter.  Remove it
%% afterwards; or at least NEVER use its underlining features.
\usepackage{soul}
\usepackage{bm}
\usepackage{datetime}

%common
\usepackage{comment}
\usepackage{ifthen}
\usepackage{todonotes}
\usepackage{titlesec}

%lists
\usepackage{listings}
\usepackage{enumerate}


%plain text
\usepackage{verbatim}


%% Some more packages that you may want to use.  Have a look at the
%% file, and consult the package docs for each.
%% See the TeXed file for more explanations

%% [OPT] Multi-rowed cells in tabulars
\usepackage{multirow}

%% [REC] Intelligent cross reference package. This allows for nice
%% combined references that include the reference and a hint to where
%% to look for it.
\usepackage{varioref}

%% [OPT] Easily changeable quotes with \enquote{Text}
%\usepackage[german=swiss]{csquotes}

%% [REC] Format dates and time depending on locale
\usepackage{datetime}

%% [OPT] Provides a \cancel{} command to stroke through mathematics.
%\usepackage{cancel}

%% [NEED] This allows for additional typesetting tools in mathmode.
%% See its excellent documentation.
\usepackage{mathtools}

%% [ADV] Conditional commands
%\usepackage{ifthen}

%% [OPT] Manual large braces or other delimiters.
%\usepackage{bigdelim, bigstrut}

%% [REC] Alternate vector arrows. Use the command \vv{} to get scaled
%% vector arrows. (package texlive-fonts-extra)
\usepackage[h]{esvect}

%% [NEED] Some extensions to tabulars and array environments.
\usepackage{array}

%% [OPT] Postscript support via pstricks graphics package. Very
%% diverse applications.
%\usepackage{pstricks,pst-all}

%% [?] This seems to allow us to define some additional counters.
%\usepackage{etex}

%% [ADV] XY-Pic to typeset some matrix-style graphics
%\usepackage[all]{xy}

%% [OPT] This is needed to generate an index at the end of the
%% document.
%\usepackage{makeidx}

%% [OPT] Fancy package for source code listings.  The template text
%% needs it for some LaTeX snippets; remove/adapt the \lstset when you
%% remove the template content.
\usepackage{listings}
\lstset{language=TeX,basicstyle={\normalfont\ttfamily}}

%% [REC] Fancy character protrusion.  Must be loaded after all fonts.
\usepackage[activate]{pdfcprot}

%% [REC] Nicer tables.  Read the excellent documentation.
\usepackage{booktabs}

%% International System measurement units (package texlive-science)
\usepackage{siunitx}

%% Subfigures
%\let\subcaption\undefined
%\let\subfloat\undefined
%\usepackage{subcaption}

%section customisation
\usepackage{titlesec}

%% Advanced figures
\usepackage{tikz}

%% Electronics circuits
%\usepackage[arrowmos]{circuitikz}

%%Image position
\usepackage{float}

%%Long tables
\usepackage{longtable}
\usepackage{tabu}

%% LaTeX' own graphics handling
\usepackage{graphicx}

%more rows
\usepackage{multirow}

%multiline equations
\usepackage{amsmath}

%% The AMS-LaTeX extensions for mathematical typesetting.  Do not
%% remove.
\usepackage{amsmath,amssymb,amsfonts,mathrsfs}

%% NTheorem is a reimplementation of the AMS Theorem package. This
%% will allow us to typeset theorems like examples, proofs and
%% similar.  Do not remove.
%% NOTE: Must be loaded AFTER amsmath, or the \qed placement will
%% break
\usepackage[amsmath,thmmarks]{ntheorem}

%math
\usepackage{array}
\usepackage{mathtools}
\usepackage{amsfonts}
\usepackage{cancel}
\usepackage{amssymb}

%different enumerations
\usepackage{enumitem}

%% Make document internal hyperlinks wherever possible. (TOC, references)
%% This MUST be loaded after varioref, which is loaded in 'extrapackages'
%% above.  We just load it last to be safe.
\usepackage[linkcolor=black,colorlinks=true,urlcolor=blue,citecolor=black,filecolor=black]{hyperref}

\input{glyphtounicode}
  \pdfgentounicode=1
\usepackage{cmap}

\usepackage{accsupp}
\usepackage{calc}
\usepackage{layouts}
\usepackage{layout}

 
\mathtoolsset{showonlyrefs}  

% Lorem ipsum
%\usepackage[]{blindtext}
\usepackage{lipsum}% dummy text
% include pdfs into the latex document
\usepackage{pdfpages}
%for landscape cheatsheet
\usepackage{pdflscape}


% Units
\usepackage{units}

% tables
\usepackage{array}
\usepackage{rotating}
\usepackage{multirow}
\usepackage{longtable}

%layout
\usepackage{multicol}
\setlength{\columnseprule}{0.4pt}


%% Our layout configuration.
%% Memoir layout setup

%% NOTE: You are strongly advised not to change any of them unless you
%% know what you are doing.  These settings strongly interact in the
%% final look of the document.

% Dependencies
\usepackage{ETHlogo}

% Chapter style redefinition
\makeatletter

%% Titlepage adjustments
\pretitle{\vspace{0pt plus 0.7fill}\begin{center}\HUGE\sffamily\bfseries}
\posttitle{\end{center}\par}
\preauthor{\par\begin{center}\let\and\\\Large\sffamily}
\postauthor{\end{center}}
\predate{\par\begin{center}\Large\sffamily}
\postdate{\end{center}}

\def\@advisors{}
\newcommand{\advisors}[1]{\def\@advisors{#1}}
\def\@department{}
\newcommand{\department}[1]{\def\@department{#1}}
\def\@thesistype{}
\newcommand{\thesistype}[1]{\def\@thesistype{#1}}

\renewcommand{\maketitlehooka}{\noindent~~~~~~~~\ETHlogo[2in]}

\renewcommand{\maketitlehookb}{\vspace{1in}%
  \par\begin{center}\Large\sffamily\@thesistype\end{center}}

\renewcommand{\maketitlehookd}{%
  \vfill\par
  \begin{flushright}
    \sffamily
    \@advisors\par
    \@department, UZH
  \end{flushright}
}

\makeatother

% This defines how theorems should look. Best leave as is.
\theoremstyle{plain}
\setlength\theorempostskipamount{0pt}

%%% Local Variables:
%%% mode: latex
%%% TeX-master: "thesis"
%%% End:


%% Theorem environments.  You will have to adapt this for a German
%% thesis.
%% Theorem-like environments

%% This can be changed according to language. You can comment out the ones you
%% don't need.

\numberwithin{equation}{chapter}

%% German theorems
%\newtheorem{satz}{Satz}[chapter]
%\newtheorem{beispiel}[satz]{Beispiel}
%\newtheorem{bemerkung}[satz]{Bemerkung}
%\newtheorem{korrolar}[satz]{Korrolar}
%\newtheorem{definition}[satz]{Definition}
%\newtheorem{lemma}[satz]{Lemma}
%\newtheorem{proposition}[satz]{Proposition}

%% English variants
\newtheorem{theorem}{Theorem}[chapter]
\newtheorem{example}[theorem]{Example}
\newtheorem{remark}[theorem]{Remark}
\newtheorem{corollary}[theorem]{Corollary}
\newtheorem{definition}[theorem]{Definition}
\newtheorem{lemma}[theorem]{Lemma}
\newtheorem{proposition}[theorem]{Proposition}

%% Proof environment with a small square as a "qed" symbol
\theoremstyle{nonumberplain}
\theorembodyfont{\normalfont}
\theoremsymbol{\ensuremath{\square}}
\newtheorem{proof}{Proof}
%\newtheorem{beweis}{Beweis}


%% Helpful macros.
%% Custom commands
%% ===============

%% Special characters for number sets, e.g. real or complex numbers.
\newcommand{\C}{\mathbb{C}}
\newcommand{\K}{\mathbb{K}}
\newcommand{\N}{\mathbb{N}}
\newcommand{\Q}{\mathbb{Q}}
\newcommand{\R}{\mathbb{R}}
\newcommand{\Z}{\mathbb{Z}}
\newcommand{\X}{\mathbb{X}}

% surrounding every content with the math environment does make the content copyable from the pdf document back into latex form. In some cases for example in captions or section titles, you will need to add \protect before the printlatex command, otherwise you get a strange error about a } too many.
%Usage:: \(\printlatex{2^i}\) or \(\pl{2^i}\) as shorthand
\newcommand*{\printlatex}[1]{%
  \BeginAccSupp{%
    ActualText=\detokenize{#1},%
    method=escape,
  }%
  #1%
  \EndAccSupp{}%
}
\newcommand{\pl}[1]{\printlatex{#1}}

\newcommand{\mc}[1]{\mathcal{#1}}

%% Special characters for Expected value |E , Variance \V, \I, Prediction error \predR
\newcommand{\E}{\mathbb{E}}
\newcommand{\V}{\mathbb{V}}
\newcommand{\I}{\mathbb{I}}
\newcommand{\predR}{\mathcal{R}}

%% Fixed/scaling delimiter examples (see mathtools documentation)
\DeclarePairedDelimiter\abs{\lvert}{\rvert}
\DeclarePairedDelimiter\norm{\lVert}{\rVert}

%% Use the alternative epsilon per default and define the old one as \oldepsilon
\let\oldepsilon\epsilon
\renewcommand{\epsilon}{\ensuremath\varepsilon}

%% Also set the alternate phi as default.
%\let\oldphi\phi
%\renewcommand{\phi}{\ensuremath{\varphi}}

%% create the signum function for mathematical formulas
\newcommand{\sgn}{\operatorname{sgn}}

\DeclareMathOperator*{\xpt}{\textit{E}}
\newcommand{\argmin}{\arg\!\min}
\newcommand{\argmax}{\arg\!\max}


%%page layout settings and listing templates etc.
%% Objects numbering
\counterwithout{figure}{section}
\counterwithout{equation}{section}
\counterwithout{section}{chapter}


% Lengths and indenting
\setlength{\textwidth}{16.5cm}
\setlength{\marginparwidth}{1.5cm}
\setlength{\parindent}{0cm}
\setlength{\parskip}{0.15cm}
\setlength{\textheight}{22cm}
\setlength{\oddsidemargin}{0cm}
\setlength{\evensidemargin}{\oddsidemargin}
\setlength{\topmargin}{0cm}
\setlength{\headheight}{0cm}
\setlength{\headsep}{0cm}

% lslisting style for python
\lstset{
	basicstyle=\ttfamily,
	breaklines=true,
	commentstyle=\color{green},
	keepspaces=true,
	keywordstyle=\color{blue},
	language=Python,
	morekeywords={off},
	showstringspaces=false,
	stringstyle=\color{purple},
	title=\lstname
}

% lslisting style for matlab
\lstset{
	basicstyle=\ttfamily,
	breaklines=true,
	commentstyle=\color{green},
	keepspaces=true,
	keywordstyle=\color{blue},
	language=Matlab,
	morekeywords={off},
	showstringspaces=false,
	stringstyle=\color{purple},
	title=\lstname
}

\pdfpagewidth=\paperwidth
\pdfpageheight=\paperheight

\expandafter\def\expandafter\normalsize\expandafter{%
    \normalsize
    \setlength\abovedisplayskip{4pt}
    \setlength\belowdisplayskip{4pt}
    \setlength\abovedisplayshortskip{4pt}
    \setlength\belowdisplayshortskip{4pt}
}

% define colors for cheatsheet
%https://en.wikibooks.org/wiki/LaTeX/Colors#Predefined_colors
\definecolor{sectionColor}{HTML}{FF7F00}
\definecolor{subsectionColor}{HTML}{EE0000}
\definecolor{subsubsectionColor}{HTML}{EE6600}


\renewcommand{\familydefault}{\sfdefault}

\title{\textbf{ZNZ Introduction to Neuroscience II} \\
       Spring 2017\\\normalsize version 1.0}

\author{
	Vanessa Leite
	\vspace{2em}
	\\Repository page: \url{https://github.com/ssinhaleite/znz-intro-to-neuroscience-II-summary}\\
	Contact \href{mailto:vrcleite@gmail.com}{vrcleite@gmail.com} if you have any questions.}
	\thesistype{The Summary of the lectures in 2017}
	\department{ZNZ - Institute of Neuroinformatics, ETH}
	\date{\today}

\begin{document}
\frontmatter


%% DO NOT CHANGE.
\begin{titlingpage}
  \calccentering{\unitlength}
  \begin{adjustwidth*}{\unitlength-24pt}{-\unitlength-24pt}
    \maketitle
  \end{adjustwidth*}
\end{titlingpage}

\mainmatter

%% This change is needed if the article option for the memoir document class
%% is used, in order to count sections (article) as if they were chapters (memoir)
\counterwithout{section}{chapter}

%% Our content

\newpage
\clearpage
\pagenumbering{roman}
\setcounter{tocdepth}{3}
\setcounter{secnumdepth}{2}
\tableofcontents

\clearpage
\pagenumbering{arabic}

\newpage

\section{Cognitive Neuroscience}
\subsection{Methods}

\subsection{Perception and Attention}
\paragraph{Slide \#1: Perception is not passive, it is a process. Description of elements of perceptual process and factors that influence what individuos perceive.}
\paragraph{Slide \#2: within them: particular perception (proprioception); meaningful experiences: we are not passive analyzers}
\paragraph{Slide \#3: When the senses are activated, starts the perceptual selection}
\paragraph{Slide \#4: enviromental stimulli: sensory depravation tank: if you do not receive stimulus, your brain creates it.}
\paragraph{Slide \#5: Selective screening: -our system eliminates some factors because they are not important for us to be aware of. -internal and external factors influencing in the perception. - cronic depression: slow/fast twich (?)}
\paragraph{Slide \#6: internal factors: - state of adaptation - constancies: color constancy: your brain starts to iluminate the enviroment making you think the color is the same in different enviroments.}
\paragraph{Slide \#7: perceptual organization}

\paragraph{Slides \#8-11: Notes: -one hand: not think perception as a passive process (sensation is the passive process) - perception is noisy and makes mistakes - very complex system.}

\paragraph{Slides \#12-14: Types of attention: 1. overt attention, 2. covert attention: related with spotlight of attention (slide \#15), you can't pay attention to many things. When you pay attention in one thing, your capacity to pay attention in others decay. 3. feature attention: shadowing tasks (slide \#16) }

\paragraph{Slide \#17: frontal lobe $\rightarrow$ saccade eyes movements (overt); parieral lobe $\rightarrow$ covert attention}
\paragraph{Slide \#18: theory about how attention works}
\paragraph{Slide \#19: one "phrase(?)" in each ear.}
\paragraph{Slides \#20-21: a lot of simple experiments can be made to evaluate attention, for instance, pop out. It is to "find" elements based on one feature, more features (conjunction) $\rightarrow$ takes longer to find it. $\rightarrow$ pop out experiments can prove sinestesia}
\paragraph{Slides \#21-22: Desimone experiments. MTA (medial temporal area) $\rightarrow$ prefered direction of motion $\uparrow$. Attention is like a weight to fire particular neurons.}
\paragraph{Slides \#23-?: a lot of slides were ignored for later reading }
\paragraph{Gorila video: innatentional blindness}

\subsection{Memory}

\subsection{Decision Making}

\subsection{Emotion}

\subsection{Body Perception}

\section{Clinical Neuroscience}

\subsection{Neurology: Ophthalmology, Otology, Epileptology and Parkinson}

\subsection{Neurology: Multiple Sclerosis, Neuromuscular, Stroke and Neuropsychology}
 
\subsection{Spinal Cord Injury}

\subsection{Epilepsy}
 
\subsection{Depression}

\subsection{Schizophrenia}

\subsection{Addiction Clinics}

\subsection{Addiction in Society}

\subsection{Neurosurgery}

%%%%%%%%%%%%%%%%%%%%%%%%%%%%%%%%%%%%%%%%%%%%%%%%%%%%%%%%%%%%%%%%%%%%%%%%%%%%%%%%%%%%%%%%%%%%%%%%%%%
\newpage


\section{Previous exams}

\subsubsection{2013}
\begin{enumerate}
\item \paragraph{FMRI}
\item \paragraph{Microglia}
\item \paragraph{Circadian Rhythms}
\item \paragraph{Motor Learning}
\item \paragraph{Parkinson}
\item \paragraph{Multiple Sclerosis}
\end{enumerate}

\subsubsection{2011}
\begin{enumerate}
\item \paragraph{ fMRI is routinely used to study the neural processes underlying behavior. Please describe all the procedures necessary for conducting and correctly interpreting an fMRI study, covering the following areas:}
\subparagraph{Data acquisition (what signals are measured in fMRI, and how?)}
\subparagraph{Data analysis (which sequential routines are necessary to detect signal changes in fMRI images, using software packages such as SPM?)}
\subparagraph{Results interpretation (what inferences about neuronal activity can be drawn from fMRI results?}

answer here.

\item \paragraph{Please list the different types of long-term memory you know of. Describe their properties in humans, group them according to involved brain structures and give examples of behavioral tests that allow to model these memory types in rodents}

\item \paragraph{Sleep regulation in physiological short and long sleepers: Explain the most important principles how sleep and wakefulness are physiologically regulated and how sleep-wake regulation may differ between habitual short and long sleepers.}

\item \paragraph{Robotic tools have played a significant role in the investigation of human motor
learning.} \subparagraph{Describe the role of internal models in human motor control and how such
models are acquired} \subparagraph{Identify three unique features of robotic systems that make them valuable tools to investigate human motor learning.} \subparagraph{Discuss how these unique features could be applied to clinical assessment and therapy of sensorimotor impairments} answer here.

\item \paragraph{The term "frontotemporal dementias" subsumes a heterogeneous group of disorders:}
\subparagraph{Please describe the clinical presentations of patients with frontotemporal dementia (major clinical syndromes, and characteristic features).} \subparagraph{Which genes/gene loci have been associated with frontotemporal dementia?} \subparagraph{Please summarize which major molecular subgroups of frontotemporal dementias can be defined and briefly discuss current knowledge and/or hypotheses on underlying pathomechanisms in the two most common subgroups.} answer here.

\item \paragraph{The maintenance of central and peripheral tolerance is the reason that autoimmune diseases are relatively rare. Please answer the following questions:} \subparagraph{How does central T cell tolerance work (which organ performs T cell education, what is negative and positive selection)?} \subparagraph{What are the mechanisms of peripheral tolerance? Remember, we discussed four of them. Please shortly recapitulate.} \subparagraph{Why does the inflammation in an MS-lesion subside after a while? What mechanisms can dampen an ongoing immune response (if you do not know,
speculate!)?} answer here.

\end{enumerate}

\subsection{All Question - topics}

\subsubsection{To organize}
\begin{enumerate}
\item \paragraph{Characteristics of sleep in mammals: Do they apply to invertebrates?}
behavioural:
Sleeping site 
Quiescence
Body posture
Elevated arousal threshold
Rapid state reversibility
Physiological:
Altered EEG
Reduced muscle tone
Reduced heart rate
Reduced respiration
Reduced body temperature
Regulatory:
Compensatory response to sleep deficit or excess sleep

\item \paragraph{Non-REM-REM sleep}

REM: rapid eye movement. EEG low amplitude, mixed frequency (more similar to wake than to deep sleep EEG). Most prominent in the morning hours.
non-REM: is subdivided into four substages 1-4 in human, deep sleep consists of stages 3 and 4. In deep sleep, he EEG contains prominent slow waves (0.5-4.5 Hz, high amplitude).
cycles: REM sleep occurs every 90-100 mins during sleep (ultradian oscillator origins in the Pons). General term to describe cyclic alternation between REM and non-REM sleep. Healthy people usually start with stage 1, then 2, 3, 4, 2, REM, 2, 3, 4, REM etc.

\item \paragraph{Sleep homeostasis and marker of sleep homeostasis on the sleep EEG}

homeostasis has been defined as the coordinated physiological processes wich maintain most of the steady states in the organism; sleep homeostasis refers to the sleep need in dependance of the time spent awake. Sleep need rises exponentially during wake and declines exponentially during sleep. According to 2-process model of sleep regulation, sleep need is additionally dependant on circadian time.
	NREM-sleep is controlled thalamocortically.
        	Marker of sleep homeostasis: slow-wave activity (power of slow waves rises in recovery sleep after sleep deprivation according to the 2-process model)
        	
\item \paragraph{Endogenous sleep-promoting components: comments}

SCN (superchiasmatic nucleus of hypothalamus)
	clock genes: transcriptional/translational process
	melatonin: built during sleep
	Thalamus: control of NREM-sleep
	Pons: Regulation of REM-sleep
	Potential homeostatic sleep-promoting agents (Experiment: if CSF from a sleep-deprived animal is transferred to a rested animal, the rested animal becomes tired $\rightarrow$ there must be an agent in the CSF that accumulates during wakefulness and makes tired): adenosine, Interleukin-1b, TNFa, GHRH, prostaglandin.

\item \paragraph{Role of thalamus-correlated rhythm in sleep: comments}

Thalamus controls the NREM sleep rhythmus $\rightarrow$ EEG activation / desactivation

\item \paragraph{Circadian pacemakers, entrainment, Zeitgeber, phas-response curve}

Pacemaker: SCN (superchiasmatic nucleus of hypothalamus)
	entrainment means that the ‘inner clock’, located in the SCN, is flexible in the way that it can adapt the phase (example: time-zone flights) and the frequency (example: bunker experiments, where one ‘day’ lasts 25 hours) of the circadian clock.
	Entrainment: via light, signalling from the eye to the SCN (possible photoreceptor: Melanopsin?). In the SCN, per transcription is activated upon light signal.
Phase-response curve: depending on the circadian time, when a light pulse is presented, the phase of the circadian clock is shifted forward or backward. If the light pulse is presented shortly before the active period has started, then the phase is advanced and if the pulse is given shortly after the active period has ended, the phase is delayed (in humans). There is one time point during night when the phase shift swiches from delayed to advanced.

\item \paragraph{Which physiological and endocrine variables in human are frequently used a phase-marker of circadian rhythm}

endocrine: melatonin, adrenal gland (adrenalin, cortison); GHRH (Growth hormone releasing hormon)
	Physiological: body temperature; activity (via activity monitor); alpha-activity in the waking EEG

\item \paragraph{What is the evidence that SCN is a circadian pacemaker}

lesion method $\rightarrow$ arythmicity
         in vitro culture of a single SCN neuron    
         SCN transplant reserves the rhythm
         in vitro SCN

\item \paragraph{Which genes (gene?) are (is) involvedin generation of circadian rhythm}

in mammals: Bmal1 is rhythmically expressed by the SCN. Clock and bmal1(basic helix-loop-helix transcription factor family) build heterodimers. These heterodimers bind to E-boxes of enhancers of the per and cry gene. Per and Cry proteins dimerize outside the nucleus and are phosphorylated. The dimers re-enter the nucleus and downregulate transcription of  clock and bmal1 (negative feedback-loop). Situation is even more complex, also containing positive feedback-loops…)
        in fruit fly Cyc/Clk heterodimers activate per/tim gene transcription. Negative feedback-loop: Per/Tim heterodimers inhibit activation of their own genes via Cyc/Clk.

%%%%%%%%
\item \paragraph{Please describe briefly:} \subparagraph{Some conceptional problems in studying depression from a neuroscientific point of view} 

Depression is a disorder of subjective feeling (translation from first person perspective to third person perspective: epistemic problem). It is very difficult to evaluate the disease scientifically (animal models that are really adequate to depression, they can’t talk to tell their feelings). The definition of psychiatry depends on social values and personal evaluation of suffering, but not depends on the organic disorder. No reliable bjective markers like genetic defects or metabolic disfunctions.

\subparagraph{Some changes of the neurobiology system in a depression state}
Change of HPA-axis (hypothalamus-pituitary-adrenal)
	The prominent mechanism by which the brain reacts to acute and chronic stress is the activation of HPA-axis $\rightarrow$ cortisol levels rise.
		Hypothalamus secretes CRH (corticotropin-releasing hormon) $\rightarrow$ pituitary (hypophysis) secrets adrenalcorticotropin (ACTH) $\rightarrow$ adrenal gland secrets cortisol

Growth hormone is reduced.
	Sleep disorders (EEG!), disturbances of appetite regulation.

different activation of brain areas (activation of medioorbital cortex and ventral anterior cingulate $\rightarrow$ limbic system activated)

%%%%%%%%%
\item \paragraph{Roles of automate as models for computation}

The computational process in neurons can be investigated in neuroinformatics via   automate models (as   compared to the structural process via neuroscience)

\item \paragraph{Automate suitable models for describing the operation of neurons and networks of neurons}

Models for automates are transferable to neurons / neuronal networks:

	Feedforward processor: input $\rightarrow$ blackbox $\rightarrow$ output (without memory)
	In neuron: input correlates to the dentritic input (sum of input signals x weights), output correlates to the axonal output (fire or not fire)

	Finite State Machine: input $\rightarrow$ black-box (which remembers the state in which it is, memory!) $\rightarrow$ output
	In neuron: neurons also can feed back information to build up memory (this model accounts only for short-time memory (seconds)…). Feedback occurs, when axonal output is networked to dendrites of the very same neuron.

	Turing Machine: input $\rightarrow$ black-box containing unbounded memory $\rightarrow$ output
	Church-Turing-Thesis: this machine is able to compute all possible computations. Philosophic question: is the brain’s memory unbounded???
	
	Universal Turing Machine: can simulate the computational process of any Turing Machine, when it knows the protocol of this machine, thus it can also simulate the computational process of a neuron… But the protocol is not known (e.g.: a bee can do computations leading to very various and complex behavior, there is no computational model that could do that with the limited recourses of a few thousand neurons that a bee needs to accomplish it)

	But: Neuronal networks have different weights for the 10 exp 14 axons/dentrite connections. This is not possible to be determined genetically (not enough resources), but dependant on the microenvironment of each neuron in the developmental process. Moreover, they can adapt to the environment by changing those weights or even establishing new connections between axons and dentrites.
	Synaptic release is additionally very versatile, can be modulated chemically, and be inhibitory or excitatory etc.
	
%%%%%%%%
\item \paragraph{Animal models of behaviour allow us to investigate the symptoms of psychiatric disorders such as depression and schizophrenia. Discuss statements, giving examples of some specific model}

You always have to ensure that the animal model is valid. There are different aspects of validity that have to be guaranteed:
Face validity: quantifiable behavior and physiology in the animal model have to be similar to the symptoms in the investigated human illness.
Construct validity: the quantifiable behavior and physiology in the animal model must be a result of the same central state as in the human patient. Theoretical rationale.
Predictive validity: close correspondence between drug actions on behavior and physiology of the animal model and the human patient.
Inter-laboratory validity
Inter-species validity

 Schizophrenia:
Impairment of working memory leads to symptoms like halluzinations: lack of references against associative memories.
	Selective attention is impaired, leading to delusions (misinterpretations), confusion of external and internal stimuli and retreatment to safety (neg. symptoms). 
Specific test: Latent Inhibition (LI) test. LI paradigm: repeated non-reinforced pre-exposure to a stimulus retards subsequent conditioning to that stimulus. This reflects the ability of ‘learning not to attend’. In animals: rats reduce LI when given amphetamine $\rightarrow$ animal model for schizophrenia. Rats that get amphetamine, avoid the box where they got shocked previously in the CAR (conditioned avoidance response) and reduce licking water in the CER (conditioned emotional response) tests compared to control animals due to their impaired LI.


	Depression:
	Animal model of learned helplessness: animals are exposed to negative stimuli and don’t get the possibility to escape. This leads to the ‘learned helplessness’ symptom, especially, if the animals are very young, which means, they give up very quickly and are not able to escape unwanted situations. Learned helplessness can be measured by the escape behavior in a two-way avoidance test. In this test, animals are placed in a shuttle box and exposed to a foot shock. They are allowed to escape to the save compartment of the shuttle box. If they get conditioned for the shock with a tone, starting shortly before the shock, animals learn to escape already at the presentation of the tone. ‘Helpless’ animals are not good in escaping compared to controls.
	Chronic mild stress: Animals are chronically exposed to mild stress like food / water deprivation for some hours, not enough space, over night illumination etc. The loss of pleasure (anhedonia) is measured with the ICSS (intra-cranial self-stimulation), the PRS (progressive reward schedule) or the sucrose preference test.
	Early life stress: Pups are stressed by separating them from mother for several hours per day etc.$\rightarrow$ anhedonia
	
%%%%%%%%
\item \paragraph{What are the physiological correlations of fMRI signal and how does the fMRI signals correlate with neuronal activities
}

The physical correlation of fMRI is neural activity, resulting in an initial (about 0.5-2s) ‘undershoot’ of the proportion of oxygenated hemoglobin, due to the consumption of oxygen for the neural activity. This leads to a reduction of the BOLD (blood oxygen level dependant) signal. Neural activity seems to mediate vasodilation (maybe through to release of NO), leading to an increase of blood flow (after 2-10s), resulting in an increase of the BOLD signal due to the better blood supply. Studies comparing BOLD data with EEG data have shown that the BOLD signal rather reflects the information uptake and processing by neurons than their spiking output measured by EEG.

\item \paragraph{What is the impact of W.I. and what information do we get from it?}

\item \paragraph{What's the significance of blood-brain-barrier in infectious disease of the CNS}

\item \paragraph{ (a) Describe the properties of a finite state machine. (b) Is a single neuron kind of  a finite state machine? Explain your answer. (c) What kinds of model (artificial neurons) do you know about?}

\item \paragraph{Please give a detailed account of the process by which prions upon entering the body reach the CNS.}

\end{enumerate}

\subsubsection{Methods}
\begin{enumerate}
\item \paragraph{Describe methods for measuring motivation, attention and memory in rodents and/or primates. In which neuropsychiatric disease are these beavioural processes disrupted?}

\item \paragraph{ fMRI is routinely used to study the neural processes underlying behavior. Please describe all the procedures necessary for conducting and correctly interpreting an fMRI study, covering the following areas:}
\subparagraph{Data acquisition (what signals are measured in fMRI, and how?)}
\subparagraph{Data analysis (which sequential routines are necessary to detect signal changes in fMRI images, using software packages such as SPM?)}
\subparagraph{Results interpretation (what inferences about neuronal activity can be drawn from fMRI results?} answer.

\end{enumerate}

\subsubsection{Memory}
\begin{enumerate}
\item \paragraph{Please list the different types of long-term memory you know of. Describe their properties in humans, group them according to involved brain structures and give examples of behavioral tests that allow to model these memory types in rodents}
\end{enumerate}

\subsubsection{Sleep}
\begin{enumerate}
\item \paragraph{Sleep regulation in physiological short and long sleepers: Explain the most important principles how sleep and wakefulness are physiologically regulated and how sleep-wake regulation may differ between habitual short and long sleepers.}

\end{enumerate}

\subsubsection{Schizophrenia}
\begin{enumerate}
\item \paragraph{Please describe shortly a neurobiological model of schizophrenia}
\end{enumerate}

	





%%%%%%%%%%%%%%%%%%%%%%%%%%%%%%%%%%%%%%%%%%%%%%%%%%%%%%%%%%%%%%%%%%%%%%%%%%%%%%%%%%%%%%%%%%%%%%%%%%%
\newpage

\section{References}
The pictures used in this summary are from the class slide sets or internet, and belong to their respective owners. In the context of the summary they are used for educational purposes only.

\subsection{Cognitive Neuroscience}
\begin{itemize}
\item Christian C. Ruff and Scott A. Huettel, Chapter 6 - Experimental Methods in Cognitive Neuroscience, In Neuroeconomics (Second Edition), edited by Paul W. Glimcher and Ernst Fehr, Academic Press, San Diego, 2014, Pages 77-108, ISBN 9780124160088, \url{http://dx.doi.org/10.1016/B978-0-12-416008-8.00006-1}
\end{itemize}

\end{document}
